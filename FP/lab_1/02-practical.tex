\chapter{Практические задания}

\textbf{1. Представить следующие списки в виде списочных ячеек.}

\includeimage
{task11} % Имя файла без расширения (файл должен быть расположен в директории inc/img/)
{f} % Обтекание (без обтекания)
{H} % Положение рисунка (см. figure из пакета float)
{0.9\textwidth} % Ширина рисунка
{Задание №1 (часть 1)} % Подпись рисунка

\includeimage
{task12} % Имя файла без расширения (файл должен быть расположен в директории inc/img/)
{f} % Обтекание (без обтекания)
{H} % Положение рисунка (см. figure из пакета float)
{0.6\textwidth} % Ширина рисунка
{Задание №1 (часть 2)} % Подпись рисунка

\textbf{2. Используя только функции CAR и CDR, написать выражения, возвращающие 1) второй; 2) третий; 3) четвертый элементы заданного списка.}

\begin{enumerate}[label=\arabic*)]
	\item (CAR (CDR '(a b c d e)))
	\item (CAR (CDR (CDR '(a b c d e))))
	\item (CAR (CDR (CDR (CDR '(a b c d e)))))
\end{enumerate}

\textbf{3. Что будет в результате вычисления выражений?}
\begin{enumerate}[label=\alph*)]
	\item (CAADR '((blue cube) (red pyramid))) \\
	(CAR (CAR (CDR '((blue cube) (red pyramid)))))\\
	(CAR (CAR '((red pyramid))))\\
	(CAR '(red pyramid))\\
	red\\
	Ответ: red
	
	\item (CDAR '((abc) (def) (ghi)))\\
	(CDR (CAR '((abc) (def) (ghi))))\\
	(CDR '(abc))\\
	Nil\\
	Ответ: Nil
	
	\item (CADR '((abc) (def) (ghi)))\\
	(CAR (CDR '((abc) (def) (ghi))))\\
	(CAR '((def) (ghi)))\\
	(def)\\
	Ответ: (def)
	
	\item (CADDR '((abc) (def) (ghi)))\\
	(CAR (CDR (CDR '((abc) (def) (ghi)))))\\
	(CAR (CDR '((def) (ghi))))\\
	(CAR '((ghi)))\\
	(ghi)\\
	Ответ: (ghi)
\end{enumerate}

\textbf{4. Напишите результат вычисления выражений и объясните как он получен.}

\begin{enumerate}
	\item (list 'Fred 'and 'Wilma) $\rightarrow$ (Fred and Wilma)
	\item (list 'Fred '(and Wilma)) $\rightarrow$ (Fred (and Wilma))
	\item (cons Nil Nil) $\rightarrow$ (Nil)
	\item (cons T Nil) $\rightarrow$ (T)
	\item (cons Nil T) $\rightarrow$ (Nil.T)
	\item (list Nil) $\rightarrow$ (Nil)
	\item (cons '(T) Nil) $\rightarrow$ ((T))
	\item (list '(one two) '(free temp)) $\rightarrow$ ((one two) (free temp))
	\item (cons 'Fred '(and Wilma)) $\rightarrow$ (Fred and Wilma)
	\item (cons 'Fred '(Wilma)) $\rightarrow$ (Fred Wilma)
	\item (list Nil Nil) $\rightarrow$ (Nil Nil)
	\item (list T Nil) $\rightarrow$ (T Nil)
	\item (list Nil T) $\rightarrow$ (Nil T)
	\item (cons T (list Nil)) $\rightarrow$ (T Nil)
	\item (list '(T) Nil) $\rightarrow$ ((T) Nil)
	\item (cons '(one two) '(free temp)) $\rightarrow$ ((one two) free temp)
\end{enumerate}

\textbf{5. Написать лямбда-выражение и соответствующую функцию. Представить результаты в виде списочных ячеек.}

\begin{enumerate}
	\item Написать функцию (f ar1 ar2 ar3 ar4), возвращающую: ((ar1 ar2) (ar3 ar4)).\\
	(defun f(ar1 ar2 ar3 ar4) (list (list ar1 ar2) (list ar3 ar4)))
	
	\item Написать функцию (f ar1 ar2), возвращающую: ((ar1) (ar2)).\\
	(defun f(ar1 ar2) (list (list ar1) (list ar2)))
	
	\item Написать функцию (f ar1), возвращающую: (((ar1))).\\
	(defun f(ar1) (list (list (list ar1))))
\end{enumerate}

\includeimage
{task5} % Имя файла без расширения (файл должен быть расположен в директории inc/img/)
{f} % Обтекание (без обтекания)
{H} % Положение рисунка (см. figure из пакета float)
{0.7\textwidth} % Ширина рисунка
{Задание №5} % Подпись рисунка
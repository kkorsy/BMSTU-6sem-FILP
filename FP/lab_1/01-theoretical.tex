\chapter{Теоретические вопросы}

\section{Элементы языка: определение, синтаксис, представление в памяти}

Вся информация (данные и программы) в Lisp представляется в виде символьных выражений --- S-выражений. По определению:
\begin{lstlisting}
	S-выражение ::= <атом> | <точечная пара>.
\end{lstlisting}

Атомы:
\begin{itemize}
	\item символы --- синтаксически --- набор литер (букв и цифр), начинающихся с буквы;
	\item специальные символы --- {T, Nil};
	\item самоопределимые атомы --- натуральные числа, дробные сичла, вещественные числа, строки --- последовательность символов, заключенная в двойные апострофы.
\end{itemize}

Списки и точечные пары (структуры) строятся из унифицированных структур --- бинарных узлов.

\begin{lstlisting}
	Точечная пара ::= (<атом>.<атом>) | (<атом>.<точечная пара>) 
					| (<точечная пара>.<атом>) 
					| (<точеченая пара>.<точечная пара>)
\end{lstlisting}

\begin{lstlisting}
	Список ::= <пустой список> | <непустой список>
		<пустой список> ::= () | Nil ,
		<непустой список> ::= (<первый элемент>.<хвост>), 
		<первый элемент> ::= <S-выражение>,
		<хвост> ::= <список>.
\end{lstlisting}

Любая структура (точечная пара или список) заключается в круглые
скобки (A.B) --- точечная пара, (А) --- список из одного элемента, пустой список изображается как Nil или ().

Непустой список по определению может быть изображен: (А.(B.(C.(D())))). Допустимо изображение списка последовательностью атомов, разделенных пробелами --- (A B C D).

Элементы списка могут, в свою очередь, быть списками (любой список заключается в круглые скобки), например --- (А (B C) (D (E))). Таким образом, синтаксически наличие скобок является признаком структуры --- списка или точечной пары.

Любая непустая структура Lisp в памяти представляется списковой ячейкой, хранящей два указателя: на голову (первый элемент) и хвост --- все остальное.

На рисунке \ref{img:theory} показано представление в памяти точечной пары (A.B) и списка из двух элементов --- (A B).

\includeimage
{theory} % Имя файла без расширения (файл должен быть расположен в директории inc/img/)
{f} % Обтекание (без обтекания)
{H} % Положение рисунка (см. figure из пакета float)
{0.5\textwidth} % Ширина рисунка
{Представление в памяти точечной пары и списка} % Подпись рисунка

\section{Особенности языка Lisp. Структура программы. Символ апостроф}

Особенности языка Lisp:
\begin{enumerate}
	\item в Lisp используется символьная обработка;
	\item программа и данные в Lisp представлены в виде списков (едины в своем физическом представлении).
	\item Lisp является бестиповым языком;
	\item память выделяется блоками. Lisp сам распределяет память.
\end{enumerate}

Символ апостроф (<<'>>) --- блокирует вычисление своего аргумента. В качестве своего значения выдает сам аргумент, не вычисляя его. Перед числами, T и Nil апостроф можно не ставить.

\section{Базис языка Lisp. Ядро языка}

Базис --- минимальный набор обозначений, к которым можно свести все правильные (вычислимые) формулы системы.
Базис Lisp образуют: атомы, структуры, базовые функции, базовые функционалы.

Ядро языка --- совокупность базиса языка и наиболее часто используемых функций.
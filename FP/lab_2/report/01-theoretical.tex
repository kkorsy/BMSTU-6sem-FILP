\chapter{Теоретические вопросы}

\section{Базис языка Lisp}

Базис --- минимальный набор обозначений, к которым можно свести все правильные (вычислимые) формулы системы.
Базис Lisp образуют:

\begin{itemize}
	\item атомы и структуры (представляющиеся бинарными узлами);
	\item базовые функции и функционалы:
	\begin{itemize}
		\item встроенные --- примитивные функции (atom, eq, cons, car, cdr);
		\item специальные функции и функционалы (quote, cond, lambda, eval, apply, funcall).
	\end{itemize}
\end{itemize}

\section{Классификация функций}

Классификация функций:
\begin{enumerate}
	\item чистые функции;
	\item специальные функции;
	\item псевдофункции;
	\item функции высших порядков (функционалы).
\end{enumerate}

По назначению функции делятся на:
\begin{enumerate}
	\item селекторы;
	\item конструкторы;
	\item предикаты.
\end{enumerate}

\section{Способы создания функций}

\begin{itemize}
	\item Функция без имени определяется с помощью \textbf{$\lambda$-выражений}. 
	Lambda-определение безымянной функции:
	\begin{center}
		\texttt{(lambda <lambda-список> <S-выражение>)}
	\end{center}
	
	Lambda-вызов функции:
	\begin{center}
		\texttt{(<lambda-выражение> <формальные параметры>)}
	\end{center}
	
	\item Функция с именем определяется с помощью \textbf{defun}:
	\begin{center}
		\texttt{(defun <имя ф-ии> (<список арг.>) <lambda-выражение>)}
	\end{center}
	
	Вызов функции:
	\begin{center}
		\texttt{(<имя ф-ии> <формальные параметры>)}
	\end{center}
	
\end{itemize}

\section{Функции Car и Cdr}

Функции \textbf{car}, \textbf{cdr} являются базовыми функциями доступа к данным. Являются чистыми функциями, могут быть применены только к структурам, принимают только 1 аргумент.

\begin{itemize}
	\item \textbf{car} обеспечивает переход по car-указателю и возвращает то, к чему был получен доступ. 
	\item \textbf{cdr} обеспечивает переход по cdr-указателю и возвращает то, к чему был получен доступ.
\end{itemize}

\section{Назначение и отличие в работе Cons и List}

Функции \textbf{list}, \textbf{cons} являются функциями создания структур. 

\begin{itemize}
	\item \textbf{cons} принимает 2 аргумента; создает списочную ячейку и устанавливает два указателя на аргументы; является чистой функцией.
	
	\item \textbf{list} принимает переменное число аргументов; возвращает список, элементы которого --- переданные в функцию аргументы; не является базисной.
	
\end{itemize} 